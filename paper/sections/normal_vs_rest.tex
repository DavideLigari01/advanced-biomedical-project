\subsection{Normal VS Abnormal VS Artifact Model}
One of the primary concerns in applying machine learning models to the medical field is the potential 
consequence of misclassifying abnormal conditions as normal. Specifically, in the case of heartbeats, 
misclassifying an abnormal heartbeat as normal can have severe clinical implications. Therefore, 
our focus shifts from achieving the highest overall accuracy to minimizing the false positive rate (FPR) 
for the normal class. This approach ensures that fewer abnormal heartbeats are incorrectly classified as 
normal, thereby enhancing patient safety and diagnostic reliability.

\subsubsection{Methodology}
To address this scenario we grouped the heartbeats into three classes: normal, abnormal, and artifact, with this
latter including extrahls, murmurs and extrastoles. We then selected some models from the previous section,
prioritizing those with a low risk score. The features used were the filtered ones and no class balancing was
necessary since the classes were balanced.
Each model was trained on 80\% of the data and tested on the remaining 20\%.


\subsubsection{Results}