\documentclass[twocolumn]{class}


\addbibresource{References.bib}

% add path to images
\graphicspath{ {./images/} }

\title{Heart Disease Recognition \\ from heart beat audio signals}
\shorttitle{Heart Disease Prediction}

\github{https://github.com/DavideLigari01/advanced-biomedical-project}

\author{Ligari D. • Alberti A.}
\university{\hspace{1.7cm}Department of Computer Engineering, Data Science, University of Pavia, Italy \newline\centering Course of Advanced Biomedical Machine Learning}
\keywords{ Heartbeat Classification • Machine Learning • Audio Features • Correlation Analysis • Ensemble Models}
\abstract{
   The early detection of heart diseases is crucial for reducing mortality, 
   yet traditional diagnostic methods often fail to identify conditions until advanced 
   stages. Leveraging advancements in machine learning, this study presents two different 
   models aimed at enhancing heart disease detection from heart sound recordings. The models, 
   MLP\_Ensemble5 and MLP\_Ensemble2, were developed using advanced ensemble techniques 
   and optimized to balance computational efficiency with diagnostic accuracy. 
   MLP\_Ensemble5 focuses on minimizing false normals, and MLP\_Ensemble2 emphasizes overall performance 
   and incorporates explainability measures to aid medical professionals. 
   The study utilizes a dataset of heart sound recordings, processed through steps 
   of data preprocessing, feature extraction using several spectral coefficients (MFCCs, Chroma STFT\dots),
   and feature selection to develop robust predictive models. Despite promising 
   results, the research faced limitations such as the small dataset size, leading to potential 
   biases due to lack of a validation set, and challenges in capturing complete cardiac 
   cycles with the chosen extraction intervals. Moreover, the models’ explainability, necessitates further validation. 
   Future work will focus on expanding the dataset, optimizing extraction intervals to 
   capture full cardiac cycles, exploring alternative feature extraction techniques, 
   and enhancing model explainability to ensure clinical applicability. This study 
   demonstrates the potential of machine learning models in heart disease detection 
   and underscores the need for further research to overcome current limitations and 
   enhance model reliability in clinical settings.
}
\firstauthor{Alberti Ligari}
\publicationdate{\today}


\begin{document}

\maketitle
\pagestyle{FirstPage}

\tableofcontents
% \nocite{dizdar_dns_2021}

% ------------------- START OF SECTIONS -------------------


% ------------------- Introduction -------------------

\section{Introduction}
\firstword{H}{eart} disease remains a leading cause of mortality worldwide.
Early diagnosis is critical for effective treatment and management.
Traditional methods of diagnosis often involve invasive procedures and expensive equipment.
Recent advancements in machine learning have opened new avenues for non-invasive diagnosis using heart sound recordings.
This paper explores the use of machine learning and network analysis to predict heart disease from heart beat audio signals.
Despite significant progress, gaps remain in accurately classifying heart sounds due to data imbalance and the presence of noise in recordings.
This study aims to address these gaps by employing advanced data preprocessing techniques and robust machine learning models.
The research question guiding this study is: How can machine learning models be optimized to improve the accuracy of heart disease prediction
from heart sound recordings?

\subsection{Problem Domain}

\subsection{Research Question}

\subsection{Previous Research}

\begin{table*}[ht!]
    \small
    \centering
    \begin{tabular}{|c|c|c|c|c|c|}
    \hline
    \textbf{Authors} & \textbf{Models} & \textbf{Features} & \textbf{Results} & \textbf{Anno} & \textbf{Dataset} \\ \hline
    W. Zhang et al \cite{Zhang_Han_Deng_2017} & SVM & Spectrogram & 0.76 Precision & 2017 & N, M, EH, AR \\ \hline
    SW. Deng et al \cite{Deng_Han_2016} & SVM & DWT & 0.76 Precision & 2016 & N, M, EH, AR \\ \hline
    A. Raza et al \cite{Raza_Mehmood_Ullah_Ahmad_Choi_On_2019} & LSTM & 1D time series & 0.80 Accuracy & 2019 & N, M, ES \\ \hline
    T. Alafif et al \cite{Alafif_Boulares_Barnawi_Alafif_Althobaiti_Alferaidi_2020} & 2D-CNN + transfer learning & MFCC & 0.89 Accuracy & 2020 & N, A \\ \hline
    Noman et al \cite{Noman_Ting_Salleh_Ombao_2019} & Ensemble CNN & 1D time series + MFCC & 0.89 Accuracy & 2019 & N, A \\ \hline
    Chen et al \cite{Chen_Ren_Hao_Hu_2018} & 2D CNN & WT + Hilbert-Huang & 0.93 Accuracy & 2018 & N, M, ES \\ \hline
    Our Model & Ensemble Model (MLPs + RF) & MFCC + Chroma + ZCR & 0.88 Accuracy & 2024 & AR, M, N, EH, ES \\ \hline
    \end{tabular}
    \caption{Comparison of different models for classification}
\end{table*} % both
\pagestyle{OtherPage}

\section{Methods}
\subsection{Source of Data}

\subsubsection{Type of sources} % Davide

\subsubsection{Classes} % Davide

\subsubsection{Data Distribution} % Davide
 % mix

To prepare the data several preprocessing operations were performed:

\vspace{0.2cm}\noindent
\textbf{Noise Reduction:} the audio data was already provided in a clipped format 
to minimize noise and irrelevant information.

\vspace{0.2cm}\noindent
\textbf{Normalization}: the audio are loaded using the \textit{torchaudio.load()} 
function, which normalized the audio signals in the range [-1, 1]. 

\vspace{0.2cm}\noindent
\textbf{Removal of Corrupted Files:} corrupted files were identified and removed 
from the dataset to ensure data quality.

\vspace{0.2cm}\noindent
\textbf{Outlier Detection and Removal:} we investigated the average duration of 
each class and found the 'artifact' class to have a significantly larger average 
duration. This was due to a few long lasting audio 
recordings (see Figure \ref{fig:DataExp_outliers_Artifacts}). A large number of samples from 
the same audio might not be as informative, thereby we used IQR to detect and remove outliers.

\vspace{0.2cm}\noindent
\textbf{Resampling:} we evaluated two sampling rates to determine the optimal rate 
for heartbeat sounds and all audio files were resampled to a common frequency of 4000 Hz 
(see Section \ref{sec:sampling_rate}).

\vspace{0.2cm}\noindent
\textbf{Segmentation:} the audio data was segmented into 1-second intervals, 
identified as the optimal extraction interval (see Section \ref{sec:extraction_interval}), as
it offered both good performance and dataset size increasing.

\vspace{0.2cm}\noindent
\textbf{Hop and Window Size}: the hop size determines the number of samples between 
successive windows, while the window size determines the number of samples considered. 
Each feature was extracted using the same window length and hop length facilitating a 
fair assessment of each feature's contribution to the classification task. 

\begin{figure}[H]
	\centering
	\includegraphics[width=1\columnwidth]{./images/DataExp_outliers_artifact.png}
	\caption{Outliers in the Artifacts class.}
	\label{fig:DataExp_outliers_Artifacts}
 \end{figure}\subsection{Data Preprocessing}
 % Andrea

\subsection{Feature Extraction}
s demonstrated by \cite{Raza_Mehmood_Ullah_Ahmad_Choi_On_2019} and \cite{Chen_Sun_Chen_Xie_Wu_Xu_2021}, MFCCs 
are highly effective features for heartbeat classification. In addition to MFCCs, 
we incorporated other features to capture various characteristics of heart sounds, enhancing the classification accuracy.
The features used are explained in the following section.

\subsubsection{Features Type}  % Andrea
\textbf{MFCC}\newline
Mel-Frequency Cepstral Coefficients (MFCCs) are representations of the short-term power spectrum of sound. 
They are derived by taking the Fourier transform of a signal, mapping the powers of the spectrum onto the mel 
scale, taking the logarithm, and then performing a discrete cosine transform. MFCCs are effective in capturing 
the timbral texture of audio and are widely used in speech and audio processing due to 
their ability to represent the envelope of the time power spectrum.
In heartbeat classification, MFFCs can reflect the different perceived quality of heart sounds,
such as the presence of murmurs or other anomalies.

\vspace{0.3cm}\noindent
\textbf{Chroma STFT}\newline
Chroma features represent the 12 different pitch classes of music (e.g., C, C\#, D, etc.). 
They are particularly useful for capturing harmonic and melodic characteristics in music. 
By mapping audio signals onto the chroma scale, these features can identify pitches regardless 
of the octave, making them useful for analyzing harmonic content in heart sounds.

\vspace{0.3cm}\noindent
\textbf{RMS}\newline
Root Mean Square (RMS) measures the magnitude of varying quantities, in this case, 
the amplitude of an audio signal. It is a straightforward way to compute the energy of 
the signal over a given time frame. RMS is useful in audio analysis for detecting volume 
changes and can help identify different types of heartbeats based on their energy levels.
For example, in a given timeframe the RMS may be altered by the presence of a murmur
with respect to a normal heart sound.

\vspace{0.3cm}\noindent
\textbf{ZCR}\newline
Zero-Crossing Rate (ZCR) is the rate at which a signal changes sign, indicating how often the signal 
crosses the zero amplitude line. It is particularly useful for detecting the noisiness and the temporal 
structure of the signal. In heartbeat classification, ZCR can help differentiate between normal and abnormal 
sounds by highlighting changes in signal periodicity.

\vspace{0.3cm}\noindent
\textbf{CQT}\newline
Constant-Q Transform (CQT) is a time-frequency representation with a logarithmic frequency scale, making it 
suitable for musical applications. Since it captures more detail at lower frequencies, it may be useful for analyzing 
the low-frequency components of heart sounds.

\vspace{0.3cm}\noindent
\textbf{Spectral Centroid}\newline
The spectral centroid indicates the center of mass of the spectrum and is often perceived as the brightness of a 
sound. It is calculated as the weighted mean of the frequencies present in the signal, with their magnitudes as 
weights. In heart sound analysis, a higher spectral centroid can indicate sharper, more pronounced sounds, 
while a lower centroid suggests smoother sounds. 

\vspace{0.3cm}\noindent
\textbf{Spectral Bandwidth}\newline
Spectral bandwidth measures the width of the spectrum around the centroid, providing an indication of the range 
of frequencies present. It is calculated as the square root of the variance of the spectrum. This feature helps 
in understanding the spread of the frequency components in the heart sounds, which can be indicative of different 
heart conditions.

\vspace{0.3cm}\noindent
\textbf{Spectral Roll-off}
Spectral roll-off is the frequency below which a certain percentage of the total spectral energy lies. It is 
typically set at 85\% and helps distinguish between harmonic and non-harmonic content. In heartbeat classification, 
spectral roll-off can be used to differentiate between sounds with a concentrated energy distribution and those with more dispersed energy.
\subsubsection{Sampling Rate Selection} % Andrea
\label{sec:sampling_rate}
The sampling rate of the data were heterogeneous, ranging from 4000 Hz to 44100 Hz, with a majority of the data being sampled at 4000 Hz.
To assess the impact of the sampling rate on the classification performance, we trained different models on
different features, extracted at different sampling rates and from various intervals. Each model is then 
evaluated using different metrics, taking into account the class imbalance issue.
We also consider a possible dependency between the sampling rate and the extraction interval, as shown 
in Algorithm \ref{alg:sr_choice}.

\begin{algorithm}
    \small
    \caption{Sampling rate choice}
    \label{alg:sr_choice}
    \begin{algorithmic}[1]
    \State \textbf{Input:}
    \State {features} = [\texttt{mfcc30 \& 120}, \texttt{cqt30 \& 70}, \texttt{chroma12}]
    \State {sampling\_rates} = [\texttt{mix}, \texttt{4000}]
    \State {extraction\_intervals} = [\texttt{0.5}, \texttt{1}, \texttt{2}, \texttt{3}]
    \State {models} = [\texttt{rf}, \texttt{svm-rbf}, \texttt{lr}]
    \State {metrics} = [\texttt{macrof1}, \texttt{mcc}]
    
    \vspace{0.2cm}
    \For{sr \textbf{in} sampling\_rates}
        \For{interval \textbf{in} extraction\_intervals}
            \For{feature \textbf{in} features}
                \State \textbf{extract} {feature} with {interval} at {sr}
                \For{model \textbf{in} models}
                    \State \textbf{train} {model} with extracted {feature}
                    \For{metric \textbf{in} metrics}
                        \State \textbf{evaluate} {model} with {metric}
                    \EndFor
                \EndFor 
            \EndFor
        \EndFor
    \EndFor
    
    \State \textbf{Output:}
    \State Given all the results, group by \texttt{model} and average the values of a specific \texttt{metric} across \texttt{features}
    
    \end{algorithmic}
    \end{algorithm}

    \noindent
The results, reported in Figure \ref{fig:FE_sampling_rate.png} showed no evident advantage to using a mix of sampling frequencies over a fixed resampled sample rate. 
Moreover, employing a fixed sample rate of 4000 Hz reduces the risk of introducing bias, enhances efficiency, 
and permits the use of a broader range of features and models.

\begin{figure}[H]
    \centering
    \includegraphics[width=1\columnwidth]{./images/FE_sampling_rate.png}
    \caption{Comparison of the macro F1 score for different sampling rates.}
    \label{fig:FE_sampling_rate.png}
\end{figure}


\subsubsection{Extraction Interval Selection}  % Andrea
\label{sec:extraction_interval}
The extraction interval refers to the duration of the audio segment from which the features are extracted.
Its choice was conducted similarly to the sampling rate selection, considering only the fixed sample rate of 4000 Hz.
However the interval also affects the number of samples available in each class, so it must be chosen carefully, 
especially considering the limited number of samples available for some classes.
The results showed that a 2-second interval yielded the best performance, however it also reduced the number of samples,
impeding a correct training and evaluation of the models. As a consequence, we picked a 1-second interval as a compromise.

\subsubsection{Number of Features per Type} % Davide % mix

\subsection{Feature Selection}
Given the large number of features (338 in total), it was necessary to identify and remove features that are poorly correlated with
the target variable as well as those that are highly correlated with each other. Due to the high number of features, a visual approach,
such as a correlation matrix, was not feasible. Instead, two filters were applied to select the most
relevant features using the Spearman correlation coefficient, as the normality test failed.\\
The first filter is based on the correlation between the features and the target variable. Features with a correlation below a certain
threshold with the target variable are removed.\\
The second filter focuses on the correlation among the features themselves.
It counts, for each feature, the number of other features with which it has a correlation above a certain threshold. Features with a
number of correlations above a specified threshold are then removed.\\

\begin{algorithm}
    \caption{Feature Selection Process}
    \begin{algorithmic}[1]
        \State \textbf{Step 1:} Compute  the normal test (D'Agostino Pearson).
        \State \textbf{Step 2:} Compute the Spearman correlation coefficient for each feature with the target variable.
        \State \textbf{Step 3:} Apply the first filter to remove features with a correlation below a certain threshold with the target variable.
        \State \textbf{Step 4:} Compute the correlation matrix among all features.
        \State \textbf{Step 5:} Apply the second filter to remove features that have a high number of correlations (above a certain threshold)
        with other features.
        \State \textbf{Step 6:} Choose threshold values empirically and apply the filters using various combinations of these thresholds.
        \State \textbf{Step 7:} Train Random Forest models on the filtered data to evaluate performance and select the best combination of thresholds.
    \end{algorithmic}
\end{algorithm}

\subsubsection*{Threshold Selection and Model Evaluation}

Threshold values were chosen empirically and the filters were applied using the combinations shown in Table \ref{tab:threshold_values}.
Using the filtered data, Random Forest models were trained and evaluated, as Random Forest was found to be the best performing model.
The optimal combination of thresholds was found to be: threshold 1 = 0, threshold 2 = 0.6 and number of features = 30, resulting in 30 features.

\rowcolors{2}{blue!8}{blue!18}
\begin{table}[h]
    \centering
    \small
    \begin{tabular}{|c|c|c|}
        \hline
        \textbf{Threshold} & \textbf{Values}                 \\
        THRESHOLD 1        & 0 - 0.1 - 0.2 - 0.3 - 0.4 - 0.5 \\
        THRESHOLD 2        & 0.6 - 0.7 - 0.8 - 0.9 - 1       \\
        N° FEATURES        & 5 - 10 - 15 - 20 - 25 - 30 - 40 \\
        \hline
    \end{tabular}
    \caption{threshold values}
    \label{tab:threshold_values}
\end{table}
\noindent
With threshold 1 = 0, the filter on the correlation between the features and the target variable was effectively bypassed.
However, with threshold 2 = 0.6, a stringent filter was applied on the correlation among the features themselves,
removing features that had a correlation above 0.6 with at least 30 other features. This indicates that having features
highly correlated with each other is more detrimental to the model than having features poorly correlated with the target variable.\\
Figure \ref{fig:comparison_model_on_all_features_vs_model_on_best} shows the results obtained with the model trained on filtered features
compared to the model trained on all features. As demonstrated, the model trained on filtered features performs significantly better.

\begin{figure}[H]
    \centering
    \includegraphics[width=0.8\columnwidth]{../images/model_on_all_features_vs_model_on_best.png}
    \caption{Comparison of different metrics between the model on all features and the model on the filtered ones}
    \label{fig:comparison_model_on_all_features_vs_model_on_best}
\end{figure}

\subsubsection*{Selected Features and Correlation Matrix}

From this analysis, 41 features remained: 28 MFCC, 12 Chroma, and 1 ZCR.
The correlation matrix of the filtered features is shown in Figure \ref{fig:correlation_matrix}.
This matrix illustrates the pairwise correlation between the selected features, with the color intensity
indicating the strength and direction of the correlation. Dark red cells represent high positive correlations,
while dark blue cells indicate high negative correlations.

\begin{figure}[H]
    \centering
    \includegraphics[width=0.8\columnwidth]{../images/correlation_matrix.png}
    \caption{Correlation matrix of the filtered features}
    \label{fig:correlation_matrix}
\end{figure}
\noindent
The matrix demonstrates that the remaining features have low correlations with each other, as evidenced by the predominantly
light colors away from the diagonal. This implies that the features are relatively uncorrelated, preventing multicollinearity
issues and enhancing the robustness of the model. The high diagonal values indicate that each feature is perfectly correlated
with itself, which is expected. However, the off-diagonal values being close to zero for most feature pairs confirm that the filtering
process was effective in selecting features that do not exhibit high inter-correlations.


 % Davide

\subsection{Models} % Davide

\subsubsection*{Metrics}

In this project, several metrics are used to evaluate the performance of the heartbeat audio classification model.
These metrics provide insights into various aspects of model performance,
particularly in the context of multiclass classification with highly imbalanced classes.

\paragraph{Accuracy}
The first metric considered is \textit{accuracy}, which is defined as the ratio of the number
of correct predictions to the total number of predictions. Mathematically, it is expressed as:
\[
    \text{Accuracy} = \frac{\text{Number of Correct Predictions}}{\text{Total Number of Predictions}}
\]
Accuracy provides a straightforward measure of overall model performance and is easy to understand and compute.
However, in the presence of class imbalance, accuracy can be misleading because it does not account for the distribution of different classes,
potentially giving a false sense of high performance when the model is biased towards the majority class.

\paragraph{Balanced Accuracy}
Balanced accuracy tries to address the limitations of accuracy in the case of imbalanced classes.
It is the average of recall obtained on each class. It is calculated as:
\[
    \text{Balanced Accuracy} = \frac{1}{N} \sum_{i=1}^{N} \frac{TP_i}{TP_i + FN_i}
\]
where \(N\) is the number of classes, \(TP_i\) is the true positives for class \(i\), and \(FN_i\)
is the false negatives for class \(i\). Balanced accuracy handles class imbalance better by giving equal weight to each class,
providing a more nuanced measure of performance across all classes.

\paragraph{Matthews Correlation Coefficient (MCC)}

The Matthews Correlation Coefficient (MCC) is a robust metric used to evaluate the performance of classification models,
especially useful in the presence of imbalanced classes. For multiclass classification problems, the MCC provides a balanced measure that takes
into account the correct and incorrect predictions across all classes. The MCC for multiclass problems is calculated as:


\[
\text{MCC} = \frac{\sum_k \sum_l \sum_m C_{kk} C_{lm} - C_{kl} C_{mk}}{\sqrt{\sum_k \left( \sum_l C_{kl} \right) \left( \sum_{k' \ne k} \sum_{l'} C_{k'l'} \right)} \sqrt{\sum_k \left( \sum_l C_{lk} \right) \left( \sum_{k' \ne k} \sum_{l'} C_{l'k'} \right)}}
\]

When there are more than two labels, the MCC will no longer range between -1 and +1. Instead, the minimum value will be between -1 and 0 depending on the true distribution. The maximum value is always +1.

This formula can be more easily understood by defining intermediate variables:

\begin{itemize}
    \item $t_k = \sum_i C_{ik}$: the number of times class $k$ truly occurred,
    \item $p_k = \sum_i C_{ki}$: the number of times class $k$ was predicted,
    \item $c = \sum_k C_{kk}$: the total number of samples correctly predicted,
    \item $s = \sum_i \sum_j C_{ij}$: the total number of samples.
\end{itemize}

This allows the formula to be expressed as:

\[
\text{MCC} = \frac{cs - \vec{t} \cdot \vec{p}}{\sqrt{s^2 - \vec{p} \cdot \vec{p}} \sqrt{s^2 - \vec{t} \cdot \vec{t}}}
\]
\noindent
The MCC ranges from -1 to 1:
\begin{itemize}
    \item An MCC of 1 indicates perfect prediction, where the model correctly identifies all instances across all classes.
    \item An MCC of 0 indicates no better performance than random chance.
    \item An MCC of -1 indicates total disagreement between predictions and actual outcomes.
\end{itemize}
\noindent
The multiclass MCC is especially valuable because it considers the distribution of errors across all classes, providing a more comprehensive
and balanced measure of performance than simpler metrics like accuracy, which can be misleading in imbalanced datasets.
By taking into account the correct and incorrect predictions for each class, the MCC ensures that performance is evaluated equitably across all classes,
making it a reliable indicator of overall model quality in multiclass classification tasks.

\paragraph{Precision and Recall}

Precision and recall are fundamental metrics used to evaluate the performance of classification models,
particularly in scenarios with imbalanced class distributions.
They provide insights into different aspects of how well a model distinguishes between classes.
Precision measures the accuracy of positive predictions. It is calculated as:
\[
    \text{Precision} = \frac{TP}{TP + FP}
\]
where \( TP \) (True Positives) is the number of correctly predicted positive instances,
and \( FP \) (False Positives) is the number of incorrectly predicted positive instances.
Precision answers the question: `Out of all instances predicted as positive, how many are actually positive'
A high precision indicates that when the model predicts a positive result, it is likely to be correct.

Recall, also known as sensitivity or true positive rate, measures the proportion of actual positives that are correctly identified by the model.
It is calculated as:
\[
    \text{Recall} = \frac{TP}{TP + FN}
\]
where \( FN \) (False Negatives) is the number of incorrectly predicted negative instances.
Recall answers the question: `Out of all actual positive instances, how many did the model correctly predict as positive'
A high recall indicates that the model is effectively capturing all positive instances, minimizing false negatives.\\
In the context of imbalanced datasets, where one class may significantly outnumber another, both precision and recall become important.\\
The micro variant aggregates these metrics across all classes, providing an overall measure that considers the total number of true positives,
false positives, and false negatives across the entire dataset.
Conversely, the macro variant computes precision and recall for each class independently and then averages them.
This method is valuable for understanding the performance of the model on individual classes, particularly when class distribution varies significantly.
It ensures that the evaluation does not disproportionately favor the majority class, thereby providing a balanced view of model performance across all classes.\\

\paragraph{F1 Score}
The F1 score is a metric that combines both precision and recall into a single value, offering a balanced assessment of a model's performance. It is defined as:
\[
    \text{F1 Score} = 2 \times \frac{\text{Precision} \times \text{Recall}}{\text{Precision} + \text{Recall}}
\]
Precision measures the accuracy of positive predictions, while recall measures the coverage of actual positives by the model.
The F1 score effectively balances these two metrics, making it particularly useful in scenarios where both precision and recall are equally important.\\
The F1 score can be computed in two ways: \textit{micro} and \textit{macro}.
The micro F1 score aggregates the contributions of all classes to compute a single F1 score.
On the other hand, the macro F1 score computes the F1 score for each class independently and then averages these scores. 
This approach is crucial in the context of
class imbalance because it ensures that the performance on minority classes receives equal weight and is not overshadowed by the majority class.

\paragraph{Receiver Operating Characteristic (ROC) Curve and Area Under the Curve (AUC)}
For binary classification scenarios, the \textit{ROC curve} and the \textit{AUC} (Area Under the Curve) are used.
The ROC curve is a plot of the true positive rate (recall) against the false positive rate, showing the trade-off between sensitivity and specificity.
The AUC provides a single value summarizing the overall ability of the model to discriminate between positive and negative classes,
with a higher AUC indicating better performance. These metrics are particularly useful for binary classification tasks,
providing a visual and quantitative assessment of model performance.

\paragraph{Risk Score} % Davide

\subsubsection{Prevention Model}
The goal of the prevention model is to provide an accessible tool for the early diagnosis of heart 
diseases, potentially usable by non-experts. Therefore, it is crucial to develop a model that 
minimizes the number of false normal predictions to accurately indicate the presence or not
of disease or identify artifacts in the provided data.

To achieve this, different heart diseases were grouped together, transforming the problem 
into a 3-class classification task: normal, disease, and artifact. Grouping the diseases not 
only simplified the classification but also balanced the class distribution. The data was 
divided into training and testing sets in an 80-20 ratio, and various models were evaluated, 
as shown in Table \ref{tab:models}.

The primary metrics for evaluating the models were the ROC-AUC score, false positive rate (FPR), 
and true positive rate (TPR), with F1-score and accuracy as secondary metrics. 
To adapt binary metrics for multi-class classification, the one-vs-rest strategy was employed. 
Specifically, we focused on the normal-vs-rest case to minimize false normal predictions.

In summary, each model was trained on the 3-class classification problem but was evaluated based on 
its binary classification performance (normal-vs-rest). The best model was selected based on its 
ROC-AUC score and performance at specific FPR levels (1\%, 5\%, 10\%, and 20\%). The objective was 
to minimize false normal predictions while maximizing true normals. A model predicting no cases 
as normal to achieve a 0\% FPR would be ineffective. % Andrea

Certainly! Here’s an improved version with enhanced argumentation:

---

\subsubsection*{Support Model}

In addition to the Prevention model, we developed a complementary model designed to assist clinicians in diagnosing heart diseases.
This model aims to accurately classify all classes present in the dataset, avoiding any simplifications, to ensure comprehensive diagnostic support.\\
Given the highly imbalanced nature of the dataset, where certain classes contain significantly fewer samples, traditional accuracy metrics can be misleading.
To address this, we explored various balancing techniques, but they did not yield satisfactory results.
Consequently, we prioritized metrics that provide equal importance to all classes, including the macro F1-score, balanced accuracy,
and Matthews Correlation Coefficient (MCC). These metrics collectively offer a robust evaluation by accounting for class imbalance,
ensuring that minority classes receive adequate consideration alongside majority classes.\\
A particular emphasis was placed on the 'Normal' class, focusing on minimizing false positives. This is crucial for patient safety,
as misclassifying abnormal heart sounds as normal could lead to missed diagnoses and delayed treatment.
To address this concern, we introduced a 'Risk score' that quantifies the impact of normal false positives, allowing
the model to be more sensitive to this critical aspect. This score was further adapted to evaluate individual diseases,
enhancing our ability to assess the model's risk for each specific condition.\\
The selection of the best model was guided by the macro F1-score and balanced accuracy.
These metrics comprehensively reflect performance across all classes, ensuring that no disease category is overlooked.
After identifying the optimal model, we employed explainability techniques to elucidate the model’s decision-making process.
 This step is crucial because the model's outputs directly impact patient care; therefore, transparency and interpretability are essential. 
 By providing clinicians with insights into how the model arrives at its conclusions, we enhance trust and facilitate informed clinical decisions.\\
To achieve this, we computed feature importance using Permutation Feature Importance, which highlights the most influential factors driving the model's predictions.\\
Additionally, we applied SHAP (SHapley Additive exPlanations) values to interpret the model's output at the individual prediction level.
 SHAP values provide a detailed breakdown of each prediction, illustrating the impact of specific features on the model's decision. 
 Based on the SHAP values, we identified the areas in the audio signal that significantly influenced the model's classification.\\
By integrating these explainability techniques, we ensure that the model’s decision-making process is transparent and comprehensible. 
This not only builds confidence among clinicians but also ensures that the model's outputs are actionable and reliable in a clinical setting. % Davide

\subsubsection{Experimented Architectures}
\rowcolors{2}{blue!8}{blue!18}
\begin{table}[h]
    \centering
    \footnotesize
    \begin{tabular}{|ll|}
        \hline
        \textbf{Name}      & \textbf{Architecture (hidden layers)}      \\ \hline
        Random Forest      & -                                          \\ 
        XGBoost            & -                                          \\ 
        CatBoost           & -                                          \\ 
        LightGBM           & -                                          \\ 
        MLP\_Basic         & (128, 64, 32)                              \\ 
        MLP\_Ultra         & (512, 256, 128, 64, 32)                    \\ 
        MLP\_Large         & (256, 128, 64, 32)                         \\ 
        MLP\_Small         & (64, 32)                                   \\ 
        MLP\_Tiny          & (32, 16)                                   \\ 
        MLP\_Reverse       & (32, 64, 128, 256, 512, 256, 128, 64, 32)  \\ 
        MLP\_Bottleneck    & (512, 64, 32)                              \\ 
        MLP\_Rollercoaster & (512, 128, 256, 128, 256, 64, 32)          \\ 
        MLP\_Hourglass     & (512, 256, 128, 64, 32, 64, 128, 256, 512) \\ 
        MLP\_Pyramid       & (1024, 512, 256, 128, 128, 128, 64, 32)    \\ 
        MLP\_Wide          & (1024, 1024)                               \\ 
        MLP\_WideUltra     & (1024, 1024, 128, 32)                      \\ 
        MLP\_Sparse        & (32, 16, 8)                                \\ 
        MLP\_Dropout       & (128, 64, 32)                              \\ 
        MLP\_Ensemble1     & MLP\_Basic, Large, Ultra                   \\ 
        MLP\_Ensemble2     & RandomForest, MLP\_Ultra                   \\ 
        MLP\_Ensemble3     & MLP\_Rollercoaster, Large                  \\ 
        MLP\_Ensemble4     & MLP\_Rollercoaster, Large, Ultra           \\ 
        MLP\_Ensemble5     & RandomForest, MLP\_Ultra, Rollercoaster    \\ 
        MLP\_Ensemble6     & MLP\_Rollercoaster, Large, Ultra, Wide     \\ 
        ALL\_Ensemble      & All models majority vote                   \\ 
        CB\_ALL\_Ensemble  & All models CatBoost                        \\ \hline
    \end{tabular}
    \caption{Models names and architectures.}
    \label{tab:models}
\end{table}

The architectures of the models used in the experiments are detailed in Table \ref{tab:models}.
Special attention is given to the ensemble models, which combine predictions from multiple models
to enhance overall performance.\\
\noindent
All MLP\_Ensemble models consist of the individual models listed in their architecture name.
These models’ predictions are combined using a soft voting strategy, where the final prediction is
determined by the argmax of the sum of the predicted probabilities from each model. This approach is
effective when the models are well-calibrated and exhibit complementary strengths and weaknesses.\\
\noindent
The ALL\_Ensemble model aggregates the predictions of all individual models using a majority vote strategy.
In contrast, the CB\_ALL\_Ensemble model also considers all individual models but uses a CatBoost model to
aggregate the predictions. This allows for a more flexible voting strategy, potentially leading to improved
performance. % Andrea


\subsection{Tools and Software}

- Scikit-learn 
- Numpy
- Pandas
- Matplotlib
- Seaborn
- Scipy
- XGBoost
- CatBoost
- PyTorch
- Torchaudio
- Librosa
- TensorFlow
- Keras
- Shap
- Imblearn
- Other Utility Libraries (e.g. joblib, os, sys, etc.) % Andrea

\section{Results}
\subsection{Prevention Model}
The initial evaluation is presented using the ROC curves of the normal class versus the others for selected models.
According to Figure \ref{fig:ROC_normVSrest_allmodels}, the MLP models outperformed the other models, as 
indicated by the higher AUC values.
Specifically, \textit{MLP\_Ensemble5} achieved the highest AUC value of 0.96, followed 
by \textit{MLP\_Ultra}, \textit{MLP\_Rollercoaster}, \textit{MLP\_Ensemble2}, 
and \textit{MLP\_Ensemble4}, all with an AUC of 0.95. \textit{MLP\_Ensemble5} also 
had the highest MCC value in the multiclass classification task, confirming its superior performance.

\begin{figure}[H]
    \centering
    \includegraphics[width=1\columnwidth]{./images/ROC_normVSrest_allmodels.png}
    \caption{ROC curves for the normal class against the rest of the classes across all models.}
    \label{fig:ROC_normVSrest_allmodels}
\end{figure}

To further analyze model performance, we selected four FPR levels (1\%, 5\%, 10\%, 20\%) and 
calculated the corresponding TPR. The consolidated results are shown in 
Figure \ref{fig:normVSrest_all}.

\begin{figure*}[htpb]
    \centering
    \includegraphics[width=1\textwidth]{./images/nomrVSrest_all.png}
    \caption{TPR at different FPR levels for all models.}
    \label{fig:normVSrest_all}
\end{figure*}

At each FPR level, \textit{MLP\_Ensemble5} outperformed the other models, 
achieving TPRs of 43.4\%, 74.3\%, 86.6\%, and 95.8\% at the 1\%, 5\%, 10\%, and 20\% FPR levels, 
respectively. Excluding \textit{MLP\_Ensemble5}, the best-performing model varied by 
FPR level: \textit{MLP\_WideUltra} at 1\%, \textit{MLP\_Ultra} at 5\%, \textit{MLP\_Ensemble2} at 10\%, 
and \textit{MLP\_Ensemble4} at 20\%.

These outcomes highlight the task's challenges in creating a model that performs well across all FPR 
levels and demonstrate the efficacy of a well-built ensemble model, which leverages the strengths of 
different models to achieve optimal performance.

\subsubsection{Best Model Analysis}
%confmat of the models composing it vs confmat of the ensemble

To further investigate the performance of the ensemble model, we compared the confusion matrices of
the individual models with the ensemble one (Figure \ref{fig:confmat_ensemble_vs_individual}).

\begin{figure}[H]
    \centering
    \includegraphics[width=1\columnwidth]{./images/confmat_ensemble_vs_individual.png}
    \caption{Confusion matrices of the individual models and the ensemble model.}
    \label{fig:confmat_ensemble_vs_individual}
\end{figure}

\noindent
In the confusion matrix the class 2 represents the normal heartbeats, the class 1 represents the 
abnormal heartbeats, and the class 0 represents the artifacts. 
We can see that the Random Forest and MLP\_Ultra are "complementary" models, with the former 
having higher true positive rates for the normal class and the latter a smaller false positive rate.
The ensemble model combines these strengths. The contribution of the MLP\_Rollercoaster model is
less evident, but it was experimentally shown that it contributes to the ensemble's performance.
The reason may be related to the fact that the some of the samples that are misclassified by the
other models are correctly classified by the MLP\_Rollercoaster model.\newline
Interestingly, the MLP\_Ultra model classifies less abnormal heartbeats as normal than the MLP\_Ensemble5 model.
However this is due to the fact that this latter simply classifies less samples as normal. Indeed we could minimize the 
FPR of the normal class just by classifying all the samples as abnormal, but this would result in a
very low TPR for the normal class.This evidence the importance of analyzing FPR and TPR together.
In conclusion, the ensemble model is the best model for the task of classifying normal heartbeats,
abnormal heartbeats, and artifacts, in a FPR/TPR trade-off maximization scenario. % Andrea

\subsection{Support Model}
The support models are evaluated using several metrics, including macro F1 score, accuracy, balanced accuracy, and MCC.
The results are depicted in Figure \ref{fig:support_models_metrics}.\\
From the figure, it is evident that accuracy tends to be higher than other metrics for all models,
indicating a potential bias as it overestimates model performance.\\
Among the models, the MLP ensembles generally show superior performance compared to individual models.
Notably, the MLP\_Ensemble2 model exhibits the highest performance, with a macro F1 score of 81.58 and an MCC of 81.53.
This suggests that ensemble models effectively leverage the strengths of individual models to achieve optimal performance.
\begin{figure}[H]
    \centering
    \includegraphics[width=.9\columnwidth]{../images/support_models_metrics.png}
    \caption{Metrics of the Support Models, computed on the test set}
    \label{fig:support_models_metrics}
\end{figure}
\noindent
The risk of misclassifying an abnormal sample as normal is depicted in Figure \ref{fig:support_models_risk_scores},
which shows the overall risk scores (in blue) for each model. The graph also displays specific risk scores associated with each class,
representing the probability of predicting a sample of that class as normal.\\
This stacked bar chart helps compare the height of the different colors rather than their areas.
From the figure, it's clear that no single model consistently outperforms others across all scores,but the performance varies across the scores.
For instance, MLP\_Rollercoaster has the best overall risk score (blue) and excels in murmurs risk score (red),
while MLP\_Ensemble3 performs best for extra systoles risk (orange).

\begin{figure}[H]
    \centering
    \includegraphics[width=\columnwidth]{../images/support_models_risk_scores.png}
    \caption{Risk Scores of the Support Models}
    \label{fig:support_models_risk_scores}
\end{figure}

\subsubsection*{Best Model}

The MLP\_Ensemble2 model stands out as the best-performing model among the support models.
To thoroughly understand its performance, we computed the confusion matrix on the test set, which is depicted in Figure \ref{fig:support_models_conf_matrix}.
The confusion matrix reveals that the model excels in recognizing artifacts (class 0) and extra systoles (class 1).
However, it tends to confuse murmurs (class 2) and extra systoles (class 4) with normal heartbeats (class 3).
\begin{figure}[H]
    \centering
    \includegraphics[width=0.8\columnwidth]{../images/support_models_conf_matrix.png}
    \caption{Confusion Matrix of the MLP\_Ensemble2 model. Columns represent the true classes, while rows represent the predicted classes. Class 0: Artifacts, Class 1: Extra heartbeats, Class 2: Murmurs, Class 3: Normal, Class 4: Extra systoles.}
    \label{fig:support_models_conf_matrix}
\end{figure}
\noindent
To further investigate this anomaly, we analyzed the mean values of the features within each class, as shown in Figure \ref{fig:mean_val_for_features}.
The mean values represent the centroids of the classes in the feature space, providing insights into the distribution of the classes.
\begin{figure}[H]
    \centering
    \includegraphics[width=\columnwidth]{../images/mean_val_for_features.png}
    \caption{Mean values for each feature within each class}
    \label{fig:mean_val_for_features}
\end{figure}
\noindent
The analysis indicates that artifacts and extra heartbeats have distinctly different mean values compared to other classes, while murmurs, extra systoles,
and normal heartbeats exhibit similar mean values. To visualize this better, we employed T-SNE to map the features into a 2D space, as shown in Figure \ref{fig:t_sne_visualization}.

\begin{figure}[H]
    \centering
    \includegraphics[width=.9\columnwidth]{../images/t-sne_feature_visualization.png}
    \caption{T-SNE visualization of the features}
    \label{fig:t_sne_visualization}
\end{figure}
\noindent
The T-SNE visualization confirms the clear separation between artifacts and extra heartbeats from the other classes, while murmurs, extra systoles,
and normal heartbeats are overlapped. This suggests that the features employed are not sufficiently distinct to differentiate normal heartbeats
from murmurs and extra systoles, explaining the model's tendency to confuse these classes.


\subsubsection*{Explainability}

To gain insights into the model's decision-making process, we computed the feature importance using permutation importance,
as shown in Figure \ref{fig:permutation_feature_importance}. The figure reveals that the most important features are the MFCCs, particularly MFCC 3, 4, 8, and 6.
The fact that the lower MFCCs are most important, suggest that the to distinguish between classes the model doesn't require fine details of the audio signal.
Surprisingly, the zero crossing rate and chroma features are less important, indicating that the model relies heavily on the MFCCs to make predictions.

\begin{figure}[H]
    \centering
    \includegraphics[width=.8\columnwidth]{../images/permutation_feature_importance.png}
    \caption{Feature Importance computed with Permutation Importance}
    \label{fig:permutation_feature_importance}
\end{figure}

To provide a more intuitive understanding of the model's decision-making process, especially for clinicians who may need an explanation of the model's decisions,
we identified the areas of the waveform most important for the model's prediction. This was done by first identifying the most important features
for the classification of a single sample. \\
Once the most important MFCCs were identified, we plotted the waveform of the sample along with these MFCCs.
By observing the values of the MFCCs, we can pinpoint the areas of the waveform that are critical for the model's decision.
Figures \ref{fig:extrahls_feature_importance} and \ref{fig:extrahls_waveform} illustrate this process for a single extra heartbeat sample. \\
For this sample, the most important features are MFCC 8, 6, and 3. The bar plot shows that MFCC 8 and 6 have negative values,
while MFCC 3 has a positive value, indicating that the significant areas of the waveform are where MFCC 3 is high and the others are low.

\begin{figure}[H]
    \centering
    \includegraphics[width=0.8\columnwidth]{../images/extrahls_feature_importance.png}
    \caption{Feature importance for a single extra heart beat sample}
    \label{fig:extrahls_feature_importance}
\end{figure}

\begin{figure}[H]
    \centering
    \includegraphics[width=0.8\columnwidth]{../images/extrahls_waveform.png}
    \caption{Waveform of an extra heart beat sample, with most important MFCCs}
    \label{fig:extrahls_waveform}
\end{figure}
 % Davide

\subsection{Other Experiments}
To further explore the classification problem, we conducted additional experiments 
involving CNN-based feature extraction, data augmentation, and a novel approach 
using a tiered ensemble model.

\subsubsection*{CNN-Based Experiments}

We conducted a series of experiments using Convolutional Neural Networks (CNNs) to explore their 
effectiveness as feature extractors for the 5-class classification problem. The CNN was
used with ImageNet weights and was not fine-tuned.

\begin{itemize}[leftmargin=*]
    \item \textbf{VGG16 with Spectral Features}: VGG16 CNN was employed as a feature extractor with 
    spectral features (MFCC, CQT, Chroma STFT, among others) used as input images. This approach yielded 
    a Macro F1 score of approximately 65\%, which is lower than the performance achieved in the primary 
    work.
    \item \textbf{VGG16 with Raw Waveform Images}: VGG16 was also used to extract features from raw 
    waveform images. Features were taken from the 5th, 4th, and 3rd convolutional layers after pooling, 
    and various classifiers (RF, SVM, MLP) were tested on these features. This method resulted in a 
    performance of around 67\%.
\end{itemize}

\subsubsection*{Data Augmentation}

We explored data augmentation techniques to address the limited and imbalanced dataset and improve
the model's generalization capabilities.

\begin{itemize}[leftmargin=*]
    \item \textbf{Noise Addition and Speed/Pitch Alteration}: We augmented the data by adding random noise (factor 0.05)
    and altering speed and pitch. However, the improvement in performance was not significant. 
    This may be due to the limited size and inherent imbalance of the dataset, as data augmentation 
    did not alter the class distribution.
    \item \textbf{Synthetic Data Generation with SMOTEN}: Synthetic data generation was applied to the 
    less represented classes using SMOTEN. Although there was an initial spike in performance metrics, 
    this was identified as bias. The model could easily distinguish the synthetically generated data from 
    the original data. The underlying issue was the limited size of the original dataset, which did not 
    provide sufficient variability for the synthetic generation algorithm to produce realistic 
    and diverse samples.
\end{itemize}

\subsubsection*{Tiered Ensemble Model}

We attempted to decompose the classification problem into two sub-problems, according to Figure \ref{fig:tiered_ensemble}.

\begin{itemize}[leftmargin=*]
    \item \textbf{Sub-problem 1: Artifact, Normal, and Abnormal Classification}: Various models were 
    tested for distinguishing between artifact, normal, and abnormal audio. The best model 
    (MLP\_Ensemble5) achieved a balanced accuracy of approximately 89.2\%.
    \item \textbf{Sub-problem 2: Disease Classification}: Different models were also applied to 
    distinguish between different diseases, achieving a balanced accuracy of more than 90.3\% with MLP\_Ensemble2.
    \item \textbf{Final Ensemble Model}: A third model (CatBoost) was used to integrate the predictions 
    from the above sub-problems and make the final classification among the five classes. This ensemble 
    approach resulted in a balanced accuracy of 80.5\%.
\end{itemize}

\begin{figure}[H]
    \centering
    \includegraphics[width=1\columnwidth]{images/tiered_ensemble.png}
    \caption{Tiered Ensemble Model}
    \label{fig:tiered_ensemble}
\end{figure}

\noindent
Despite the promising results, the tiered ensemble model did not outperform the primary models.
 % Andrea

% Section discussion
\section{Discussion}

We developed two ensemble models, MLP\_Ensemble5 and MLP\_Ensemble2, to address the two folded 
purpose of assessing prevention and diagnosis (support model) of heart disease. In the former, we focused 
on the minimization of false negatives, while in the latter, we aimed to maximize the 
overall predictive performance, to then add explainability to the model.
In both cases we tried several models and features, taking care of correlation aspects, 
to find the best results. We used the PASCAL challenge dataset, handling the class imbalance
problem aggregating diseases in one case and using proper metrics in the other.
We added explainability to the support model, to provide insights to the medical staff.

Several studies have been conducted on the prediction of heart disease using machine learning. 
Zhang et al. \cite{Zhang_Han_Deng_2017} used a Support Vector Machine (SVM) model with spectrogram features to achieve a precision of 0.76,
while Deng et al. \cite{Deng_Han_2016} used SVM with Discrete Wavelet Transform (DWT) features to achieve a similar precision. In both
cases they used a model which is not computationally intensive, but with a limited performance.
Some advancements have been made using deep learning models, such as the Long Short-Term Memory (LSTM) model used by Raza et al. \cite{Raza_Mehmood_Ullah_Ahmad_Choi_On_2019}
with 1D time series features, achieving an accuracy of 0.80 with Normal, Murmur and Extrasystole classes.
A significant improvement in the accuracy was achieved using CNNs by Alafif et al. \cite{Alafif_Boulares_Barnawi_Alafif_Althobaiti_Alferaidi_2020} and Noman et al. \cite{Noman_Ting_Salleh_Ombao_2019},
however they only considered Normal and Abnormal classes, and the computational cost is higher. 
A more disease specific approach was taken by Chen et al. \cite{Chen_Ren_Hao_Hu_2018}, who used a 2D CNN model with Wavelet Transform and Hilbert-Huang features, achieving an accuracy of 0.93 with Normal, Murmur and Extrasystole classes.
This result may be investigated to improve the performance of our support model, which showed 
a weakness in distinguishing between Murmur, Normal and Extrasystole classes. 
In general a direct comparison is not possible, as in many cases the dataset used in the researches is different from ours, and the classes are not the same, with 
our model being the only one considering 5 classes.
A further thing to consider is the metric used to evaluate the performance. Indeed when dealing 
with imbalanced datasets, the accuracy is not a good metric, as it may be misleading.
In conclusion this research contributes to the literature by providing 
two models. The first one, MLP\_Ensemble5, was optimized to minimize false normals while keeping
a reduced computational cost and caring about distinguishing not only the presence or not of a disease but also
whether the recorded signal is artifact or not. The second one, MLP\_Ensemble2, was optimized to maximize the overall performance,
and introduced explainability measures to provide insights to the medical staff.
The projects also suffered from some limitations. Two are realated to the limited size of the dataset. 
The first one is the unability to create a validation set, thus the results may be biases. The second 
one is that to increase the available data we were forced to select a one second extraction interval, which may
not be optimal, as people may have different heart rates, also below the 60 bpm. Thereby by doing that 
we may have samples with no information, as the heart sound may be not present in the interval.
Another limitation we found is in the used features. We found MFCCs, while working well, to not be as discriminative
for the distinction between Normal, Murmur and Extrasystole classes, which is where the model suffered the most.
Lastly we have have to consider limits in the explainability of the model. 
Indeed we showed the top MFCCs variationalong the waveform, idnetifying the points where the values may be
more relevant, according to the SHAP results. However the model is not fed with the MFCC time series,
but it relies on their mean across time. As a consequence the interpretation of the results may be done with 
extreme caution. 

There are then obvious limitations in the dataset representative power, which is the main 
obstacle in the development of a model that can be used in real world scenarios.

% Section Conclusion
\section{Conclusion}
\subsection{Overall Impression}

\subsection{Future Work}

\clearpage
\section{Appendix}

% Figure on the best prevention model


% color the image in black
\begin{figure}[H]
    \centering
    \includegraphics[width=1\columnwidth]{./images/MLP_Ensemble5.png}
    \caption{MLP Ensemble5 Architecture}
    \label{fig:MLP_Ensemble5}
\end{figure}

\clearpage
\printbibliography

\end{document}