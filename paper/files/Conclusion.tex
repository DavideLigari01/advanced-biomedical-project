\section{Conclusion}
In this project, we have developed a machine learning model aimed at improving heart disease detection.
Utilizing advanced ensemble techniques, we optimized two models : MLP\_Ensemble5 and MLP\_Ensemble2.
The MLP\_Ensemble5 model was specifically fine-tuned to minimize false normals while maintaining low computational costs.
On the other hand, MLP\_Ensemble2 focused on overall performance and incorporated explainability measures to aid medical professionals in decision-making.\\
Despite the notable advancements, our project faced several limitations. The size of the dataset constrained our ability to create a validation set,
which might have introduced biases in the results. Additionally, the one-second extraction interval used
to increase data availability could have resulted in missing critical information from complete cardiac cycles.
Feature selection posed another challenge, as Mel Frequency Cepstral Coefficients (MFCCs), while effective,
were not sufficiently discriminative for certain heart conditions like Murmur and Extrasystole.\\
Moreover, the model's explainability, while a significant feature, had limitations.
The reliance on average temporal values rather than instantaneous ones meant that the explanations provided by the model were not always fully
reliable and required further validation.

\vfill
\subsection{Future Works}
Future research should address the limitations identified in this project to enhance the model's applicability and reliability in real-world clinical settings.
Specifically, expanding the dataset and improving its representativeness will be crucial. A larger and more diverse dataset will capture a wider
range of heart sound variations, improving the model's ability to generalize across different patient populations and conditions.
Collaborating with multiple healthcare institutions to collect more comprehensive data, which includes diverse demographic and clinical characteristics, will be essential.\\
Optimizing the extraction intervals to capture complete cardiac cycles could lead to better model performance.
The current one-second intervals may miss critical information. By adjusting extraction intervals to encompass entire cardiac cycles,
the model will analyze more comprehensive heartbeat patterns, enhancing its accuracy in detecting subtle abnormalities.\\
Exploring alternative feature extraction techniques is another important area.
While MFCCs have been effective, they may not capture all relevant features of heart sounds.
Methods such as wavelet transforms or more complex neural features derived from deep learning models, such as CNNs or RNNs,
could also be explored to enhance feature extraction and model performance.\\
Finally, enhancing the model's explainability will be essential for its clinical adoption.
Collaborating with medical professionals to analyze the results of the explainability procedure will ensure the medical relevance of the identified waveform zones.