\subsection{Prevention Model}
The initial evaluation is presented using the ROC curves of the normal class versus the others for selected models.
According to Figure \ref{fig:ROC_normVSrest_allmodels}, the MLP models outperformed the other models, as 
indicated by the higher AUC values.
Specifically, \textit{MLP\_Ensemble5} achieved the highest AUC value of 0.96, followed 
by \textit{MLP\_Ultra}, \textit{MLP\_Rollercoaster}, \textit{MLP\_Ensemble2}, 
and \textit{MLP\_Ensemble4}, all with an AUC of 0.95. \textit{MLP\_Ensemble5} also 
had the highest MCC value in the multiclass classification task, confirming its superior performance.

\begin{figure}[H]
    \centering
    \includegraphics[width=1\columnwidth]{./images/ROC_normVSrest_allmodels.png}
    \caption{ROC curves for the normal class against the rest of the classes across all models.}
    \label{fig:ROC_normVSrest_allmodels}
\end{figure}

To further analyze model performance, we selected four FPR levels (1\%, 5\%, 10\%, 20\%) and 
calculated the corresponding TPR. The consolidated results are shown in 
Figure \ref{fig:normVSrest_all}.

\begin{figure*}[htpb]
    \centering
    \includegraphics[width=1\textwidth]{./images/nomrVSrest_all.png}
    \caption{TPR at different FPR levels for all models.}
    \label{fig:normVSrest_all}
\end{figure*}

At each FPR level, \textit{MLP\_Ensemble5} outperformed the other models, 
achieving TPRs of 43.4\%, 74.3\%, 86.6\%, and 95.8\% at the 1\%, 5\%, 10\%, and 20\% FPR levels, 
respectively. Excluding \textit{MLP\_Ensemble5}, the best-performing model varied by 
FPR level: \textit{MLP\_WideUltra} at 1\%, \textit{MLP\_Ultra} at 5\%, \textit{MLP\_Ensemble2} at 10\%, 
and \textit{MLP\_Ensemble4} at 20\%.

These outcomes highlight the task's challenges in creating a model that performs well across all FPR 
levels and demonstrate the efficacy of a well-built ensemble model, which leverages the strengths of 
different models to achieve optimal performance.

\subsubsection{Best Model Analysis}
%confmat of the models composing it vs confmat of the ensemble

To further investigate the performance of the ensemble model, we compared the confusion matrices of
the individual models with the ensemble one (Figure \ref{fig:confmat_ensemble_vs_individual}).

\begin{figure}[H]
    \centering
    \includegraphics[width=1\columnwidth]{./images/confmat_ensemble_vs_individual.png}
    \caption{Confusion matrices of the individual models and the ensemble model.}
    \label{fig:confmat_ensemble_vs_individual}
\end{figure}

\noindent
In the confusion matrix the class 2 represents the normal heartbeats, the class 1 represents the 
abnormal heartbeats, and the class 0 represents the artifacts. 
We can see that the Random Forest and MLP\_Ultra are "complementary" models, with the former 
having higher true positive rates for the normal class and the latter a smaller false positive rate.
The ensemble model combines these strengths. The contribution of the MLP\_Rollercoaster model is
less evident, but it was experimentally shown that it contributes to the ensemble's performance.
The reason may be related to the fact that the some of the samples that are misclassified by the
other models are correctly classified by the MLP\_Rollercoaster model.\newline
Interestingly, the MLP\_Ultra model classifies less abnormal heartbeats as normal than the MLP\_Ensemble5 model.
However this is due to the fact that this latter simply classifies less samples as normal. Indeed we could minimize the 
FPR of the normal class just by classifying all the samples as abnormal, but this would result in a
very low TPR for the normal class.This evidence the importance of analyzing FPR and TPR together.
In conclusion, the ensemble model is the best model for the task of classifying normal heartbeats,
abnormal heartbeats, and artifacts, in a FPR/TPR trade-off maximization scenario.