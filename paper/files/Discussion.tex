\section{Discussion}

In our research, we developed two ensemble models, MLP\_Ensemble5 and MLP\_Ensemble2, to address the dual purpose of enhancing prevention and diagnosis
(support model) of heart disease. For MLP\_Ensemble5, our focus was on minimizing false normals, ensuring that potential heart diseases are not overlooked.
On the other hand, MLP\_Ensemble2 aimed to maximize overall predictive performance while incorporating explainability
to provide valuable insights to medical professionals.\\
Our experimentation involved various models and features, with careful consideration of correlation aspects,
to achieve the best results. We utilized the PASCAL challenge dataset and addressed the class imbalance problem by aggregating diseases
in one instance and employing appropriate metrics in another.\\
Several studies have explored heart disease prediction using machine learning. Zhang et al. \cite{Zhang_Han_Deng_2017}
utilized a Support Vector Machine (SVM) model with spectrogram features, achieving a precision of 0.76, while Deng et al. \cite{Deng_Han_2016}
used SVM with Discrete Wavelet Transform (DWT) features, obtaining similar precision.
Although these models are computationally efficient, their performance is limited.
Advancements in deep learning, such as the Long Short-Term Memory (LSTM) model used by Raza et al. \cite{Raza_Mehmood_Ullah_Ahmad_Choi_On_2019}
with 1D time series features, achieved an accuracy of 0.80 with Normal, Murmur, and Extrasystole classes.\\
Significant improvements in accuracy were achieved using Convolutional Neural Networks (CNNs)
by Alafif et al. \cite{Alafif_Boulares_Barnawi_Alafif_Althobaiti_Alferaidi_2020} and Noman et al. \cite{Noman_Ting_Salleh_Ombao_2019},
although they only considered Normal and Abnormal classes. A more disease-specific approach was taken
by Chen et al. \cite{Chen_Ren_Hao_Hu_2018}, who used a 2D CNN model with Wavelet Transform and Hilbert-Huang features, achieving an
accuracy of 0.93 with Normal, Murmur, and Extrasystole classes. This high accuracy highlights the potential for
improving our support model, which struggles to distinguish between Murmur, Normal, and Extrasystole classes.\\
Direct comparisons between studies are challenging due to differences in datasets and class definitions.
Our model uniquely considers five classes, making it distinct from others. Furthermore, the choice of evaluation metrics is crucial when dealing
with imbalanced datasets, as accuracy can be misleading.\\
Our research contributes to the literature by presenting two models. MLP\_Ensemble5 was optimized to minimize
false normals while maintaining low computational costs and distinguishing between disease presence and artifact signals.
MLP\_Ensemble2 was optimized for overall performance and introduced explainability measures to assist medical staff.\\
Despite these advancements, our project faced limitations. The limited dataset size prevented the creation of a validation set, potentially biasing results.
Additionally, to increase data availability, we selected a one-second extraction interval, which may not be optimal due to varying heart rates.
This approach might result in samples lacking relevant information if the full cardiac cycle is not present within the interval.\\
Feature selection also posed challenges. While MFCCs proved effective, they were not sufficiently discriminative for distinguishing between Normal,
Murmur, and Extrasystole classes, where our model struggled the most. Furthermore, the explainability of the model has its constraints.
We identified significant temporal regions on the waveform based on MFCC values within these regions. 
According to SHAP values, these specific MFCC values help to classify the sample correctly.
 However, the model only understands the average temporal values, not the instantaneous ones. 
 Consequently, the explanation provided by the model is limited and requires further validation to be considered reliable.\\
Finally, the representativeness of the dataset remains a major obstacle to developing a model suitable for real-world scenarios.
Addressing these limitations in future research could enhance the applicability and reliability of our models in clinical settings.
