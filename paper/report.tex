\documentclass[twocolumn]{class}


\addbibresource{References.bib}

% add path to images
\graphicspath{ {./images/} }

\title{Heart Disease Prediction from heart beat audio signals using Machine Learning and Network Analysis}
\shorttitle{Heart Disease Prediction}

\github{https://github.com/DavideLigari01/advanced-biomedical-project}

\author{Ligari D. • Alberti A.  }

\affil[1]{Department of Computer Engineering, Data Science, University of Pavia, Italy \newline\centering Course of Advanced Biomedical Machine Learning}

\keywords{---TO BE DEFINED---}
\abstract{  
   ---TO BE DEFINED---
}
\firstauthor{Alberti Ligari}
\publicationdate{\today}


\begin{document}

\maketitle
\pagestyle{FirstPage}

\tableofcontents
% \nocite{dizdar_dns_2021}

% ------------------- START OF SECTIONS -------------------


% ------------------- Introduction -------------------

% Introduction to the medical problem and how it is currently addressed (some bibliography)
\input{./sections/introduction.tex}
\pagestyle{OtherPage}

% ------------------- Diseases -------------------


\section{Dataset Description}

The dataset for this project was obtained from a Kaggle repository titled \textit{Dangerous Heartbeat Dataset (DHD)} \cite{Dangerous-Heartbeat-Dataset-DHD}, 
which in turn sources its data from the PASCAL Classifying Heart Sounds Challenge 2011 (CHSC2011) \cite{pascal-chsc-2011}. 
This dataset comprises audio recordings of heartbeats, categorized into different types of heart sounds.
Specifically, the dataset consists of 5 types of recordings: Normal Heart Sounds, Murmur Sounds, Extra Heart Sounds, Extrasystole Sounds, and Artifacts.
Data has been gathered from the general public via the iStethoscope Pro iPhone app and from a clinic trial in hospitals using the digital stethoscope DigiScope.

\subsection{Data Exploration}


\begin{figure}[H]
    \centering
    \includegraphics[width=1\columnwidth]{./images/DataExp_num_durations.png}
    \caption{Number of samples per duration.}
    \label{fig:DataExp_num_durations}
\end{figure}

\begin{figure}[H]
    \centering
    \includegraphics[width=1\columnwidth]{./images/DataExp_sr_distribution.png}
    \caption{Distribution of sampling rates.}
    \label{fig:DataExp_sr_distribution}
\end{figure}

\subsection{Data Preprocessing}

\begin{figure}[H]
	\centering
	\includegraphics[width=1\columnwidth]{./images/DataExp_outliers_artifact.png}
	\caption{Outliers in the Artifacts class.}
	\label{fig:DataExp_outliers_Artifacts}
 \end{figure}

% ------------------- Dataset -------------------

% Description of the dataset and its features (some plots and one hot encoding)
\section{Dataset Description}

The dataset for this project was obtained from a Kaggle repository titled \textit{Dangerous Heartbeat Dataset (DHD)} \cite{Dangerous-Heartbeat-Dataset-DHD}, 
which in turn sources its data from the PASCAL Classifying Heart Sounds Challenge 2011 (CHSC2011) \cite{pascal-chsc-2011}. 
This dataset comprises audio recordings of heartbeats, categorized into different types of heart sounds.
Specifically, the dataset consists of 5 types of recordings: Normal Heart Sounds, Murmur Sounds, Extra Heart Sounds, Extrasystole Sounds, and Artifacts.
Data has been gathered from the general public via the iStethoscope Pro iPhone app and from a clinic trial in hospitals using the digital stethoscope DigiScope.

\subsection{Data Exploration}


\begin{figure}[H]
    \centering
    \includegraphics[width=1\columnwidth]{./images/DataExp_num_durations.png}
    \caption{Number of samples per duration.}
    \label{fig:DataExp_num_durations}
\end{figure}

\begin{figure}[H]
    \centering
    \includegraphics[width=1\columnwidth]{./images/DataExp_sr_distribution.png}
    \caption{Distribution of sampling rates.}
    \label{fig:DataExp_sr_distribution}
\end{figure}

\subsection{Data Preprocessing}

\begin{figure}[H]
	\centering
	\includegraphics[width=1\columnwidth]{./images/DataExp_outliers_artifact.png}
	\caption{Outliers in the Artifacts class.}
	\label{fig:DataExp_outliers_Artifacts}
 \end{figure}


% ------------------- Goals -------------------

% Description of the goals of the project:
\input{./sections/goals.tex}
% - ML model to predict the disease
% - Network analysis to find the most important symptoms to reduce complexity, and enhance the available features


% -------------------Feature extraction -------------------

\section{Feature Extraction}

\subsection{Sample rate and interval selection}

\subsection{Results}



% ------------------- Feature Selection -------------------
\section{Feature Selection}

\subsection{Optimal number of features for each type of feature}


\subsection{Balancing method selection}

% ------------------- Feature Reduction -------------------
\input{./sections/feature_reduction.tex}

% ------------------- Model Selection -------------------
\section{Model Selection}

\subsection{Explanation of the models}
\subsection{Methodology}
\subsection{Results}

% ------------------- Best Model Analysis -------------------
\section{Analysis of the Best Model}

% ---------------------- Conclusion -----------------------------

\input{./sections/conclusion.tex}
\input{./sections/future_works.tex}
% - our approach in feature selection and parameters search is not optimal 
% - list the possible issues (e.g. we are training model on diverse features with the optimal parameters for other features)
% - exploit the knowledge of symptoms communities and their most pointed diseases, using it at prior probability for the model classification

\clearpage
\section{Appendix}

% Figure on the best prevention model


% color the image in black
\begin{figure}[H]
    \centering
    \includegraphics[width=1\columnwidth]{./images/MLP_Ensemble5.png}
    \caption{MLP Ensemble5 Architecture}
    \label{fig:MLP_Ensemble5}
\end{figure}

\clearpage
\printbibliography

\end{document}